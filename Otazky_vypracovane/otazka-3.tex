\section{Jaké jsou základní rozdíly mezi autorskoprávní a patentovou ochranou? Lze chránit v ČR patentem software jako takový? Jaké jsou podmínky patentovatelnosti vynálezu realizovaného počítačem (computer implemented invention)?}

\textbf{Autorské právo - } chráni len konkrétne vyjadrenie, konkrétny zdrojový kód (v prípade PC programu), nechráni funkcionalitu resp. spôsob riešenia istého technického problému. 

\textbf{Patent - } chráni spôsob interakcie jednotlivých softwarových a hardwareových prvkov za účelom dosiahnutie istého výsledku. Nechráni originálnie "dielo". Patent je spôsob riešenia istého technického problému. Patentovanie do istej miery rieši problém toho, že program je chránený ako lit. dielo (len jeho samotný kód) a snaźí sa chrániť aj samotné prevedenie nápadu.

Medzi pojmami software a počítačový program je rozdiel. Pod pojmom \textbf{software} sa rozumie obecne \textbf{všetko čo nie je hardware}. Zatiaľčo \textbf{počítačový program} je len druhom sofrware, ktorého hlavnou podstatou sú \textbf{inštrukcie} pre PC. Príkladom môže byť film, prehrávaný na PC. Podľa tejto definície sa jedná o software, avšak nie je to počítačovú program (jedná sa o audiovizuálne dielo), zatiaľčo prehrávač, na ktorom je tento film prehrávaný (napr. VLC) je už počítačovým programom.

V Európe (nie EU) sa patentovaním zaoberá \textbf{European Patent Office} (EPO). Po podaní žiadosti a úspešnom absolbovaní vyšetrovania patentovateľnosti je možné získať na svoj vynález patent. To platí pre každého z krajín tzv. úmluvy EPC z roku 1973. EPO obsahuje tzv. \textbf{boards of appeal} (BoA), ktoré prejednávajú sťažnosti voči rozhodnutiam niektorého z oddelení EPO (prijímacie, rešeršné, prieskumové, právne). \textbf{EboA - Enlarged Board of Appeal - } je nadriadený BoA, ktorý sa povoláva vo vynimočných prípadoch aby riešil dôležité otázky práva a zaisťoval jednotu jeho aplikácie. Napr. bol povolaný v prípade \href{https://www.epo.org/law-practice/case-law-appeals/recent/t140489eu2.html}{T 0489/14} kde riešil, či metóda simulácie chodcov vo verejných priestoroch je patentovateľná.

\textbf{Patentovateľnosť počítačových programov - } podľa \href{https://www.zakonyprolidi.cz/cs/1990-527#:~:text=c)%20plany%2C%20pravidla%20a%20zpusoby%20vykonavani%20dusevni%20cinnosti%2C%20hrani%20her%20nebo%20vykonavani%20obchodni%20cinnosti%2C%20jakoz%20i%20programy%20pocitacu%3B}{§3 odseku 2 zákonu č. 527/1990 Sb.} (pochádzajúceho z úmluvy EPC) nie je možné program počítača patentovať. \textbf{Avšak to platí len pre počítačový program ako taký.} Počítačový program ako súčasť spôsobu je patentovaťeľný. Zariadenia obsahujúce počítačový program ako súčasť riešenia technického problému je taktiež patentovateľné. Inak povedané, musí sa jednať o tzv. \textbf{vynález relizovaný počítačom.}

\textbf{Vynález - } právne nedevinovaný (vidím a viem). Môže to byť napr. Metóda uskutočnená počítačom, špecifické zariadenie resp. PC naprogramované k uskutočnení metódy, a pod. 

\textbf{Podmienky patentovateľnosti:}
\begin{itemize}
    \item \textbf{Nový -} musí sa jednať o vynález prinášajúci pokrok v istej oblasti
    \item \textbf{Priemyslovo využiteľný -} vynález musí byť využiteľný v hociakej oblasti priemyslu
    \item \textbf{Výsledok vynálezeckej činnosti -} nesmie vyplívať zrejmým spôsobom zo stavu techniky
\end{itemize}
