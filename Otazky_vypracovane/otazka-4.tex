\section{Jak lze chránit právem v ČR data a databáze? Jak Soudní dvůr Evropské unie vykládá pojem "podstatný vklad do pořízení, ověření nebo předvedení obsahu databáze" ve vztahu k přiznání ochrany zvláštním právem pořizovatele databáze?}

\textbf{Databáza \--- }\textit{zbierka nezávislých diel, údajov alebo iných nezávislých materiálov systematicky alebo metodicky usporiadaných a individuálne prístupných elektronickými alebo inými prostriedkami.} Podľa \href{https://eur-lex.europa.eu/legal-content/SK/ALL/?uri=celex:31996L0009}{Smernice 96/9/ES}

Databázu možno chrániť podľa \textbf{autorského práva} a \textbf{práva sui-generis}

\textit{...sú databázy, ktoré predstavujú spôsobom výberu alebo usporiadaním ich obsahov
autorov vlastný duševný výtvor, chránené ako také podľa
\textbf{autorského práva}. Žiadne iné kritériá sa nebudú uplatňovať
pri rozhodovaní o vhodnosti takejto ochrany.} Podľa \href{https://eur-lex.europa.eu/legal-content/SK/ALL/?uri=celex:31996L0009}{Smernice 96/9/ES}

\href{https://www.zakonyprolidi.cz/cs/2000-121#:~:text=zvlastni%20prava%20k%20databazi%20(%C2%A7%2090)%20prislusi%20porizovateli%20databaze%2C%20pokud%20porizeni%2C%20overeni%20nebo%20predvedeni%20obsahu%20databaze%20predstavuje%20kvalitativne%20nebo%20kvantitativne%20podstatny%20vklad%20bez%20ohledu%20na%20to%2C%20zda%20databaze%20nebo%20jeji%20obsah%20jsou%20predmetem%20autorskopravni%20nebo%20jine%20ochrany.}{§ 88a a ďalej zákona č. 121/2000 Sb.} stanovuje právo zriaďovateľa databázy zabrániť \textbf{vyťažovaniu} a/alebo \textbf{zúžitkovaniu} obsahu databáze v prípade že poriadenie, overenie alebo prevedenia obsahu databáze predstavuje podstatný vklad. \--- \textbf{právo sui-generis}

Vkladom sa nemyslí \textbf{investícia do tvorby dát resp. databáze.} Vklad musí smerovať k poriadeniu, overeniu alebo prevedeniu už existujúcich prvkov a \textbf{nesmie byť nejáko závislý na prostriedkoch vynaložených na vytvorenie dát.} (Polčák a kol., 2018)

Vklad nemusí byť len finnančný, môže sa jednať aj o \textbf{ľudské alebo technické prostriedky.} Vklad je podstatý ak nie je vklad nepodstatný a ľahko prevediteľný kýmkoľvek. Taktiež je nutné povedať, že je na osobe, ktorá tieto práva uplantňuje, aby preukázala existenciu podstatného vkladu. (Polčák a kol., 2018)

Sui-generis právo trvajú 15 rokov od dokončenia databáze. V prípade, že databáza je akýmkoľvek spôsobom sprístupnená verejnosti, zaniká sui-generis právo 15 rokov od prvého sprístupnenia. Podľa \href{https://www.zakonyprolidi.cz/cs/2000-121#:~:text=pr%C3%A1vo%20po%C5%99izovatele%20datab%C3%A1ze-,trva%2015%20let%20,-od%20po%C5%99%C3%ADzen%C3%AD%20datab%C3%A1ze}{§ 93 zákonu 121/2000 Sb.}

\textbf{Vyťažovanie \--- } trvalý alebo dočasný prenos celého obsahu databáze alebo jeho podstatnej časti na iný podklad (resp. nosič) a to akýmikoľvek prostriedkami a spôsobom.

\textbf{Zúžitkovanie \--- } akýkoľvek spôsob zprístupnenia verejnosti celého obsahu databáze alebo jeho podstatnej časti rozširovnaím, rozmnožením, prenájmom, spojením on-line alebo inými spôsobmi prenosu.

\subsection{Výnimky z autorského a sui-generis práva}

\textbf{Pre osobnú potrebu \---} \href{https://www.zakonyprolidi.cz/cs/2000-121#:~:text=uziti%20pocitacoveho%20programu%20ci%20elektronicke%20databaze%20i%20pro%20osobni%20potrebu}{§ 30 odstavec 3 zákonu 121/2000 Sb.} hovorí, že aj databázu je možné použiť pre osobnú potrebu a vytvoriť kópiu pre osobnú potrebu.

\textbf{Obmedzenia práva autorského k dielu súbornému \---} \href{https://www.zakonyprolidi.cz/cs/2000-121#:~:text=k%20d%C3%ADlu%20souborn%C3%A9mu%2C-,ktere%20je%20databazi,-%2C%20nezasahuje%20opr%C3%A1vn%C4%9Bn%C3%BD%20u%C5%BEivatel}{§ 36 zákonu 121/2000 Sb.}

\textbf{Nechránené databázy (AZ a/alebo sui-generis) môžu byť stále chránené relatívnimi majetkovími právami, ktoré vznikajú zo zmluvy, jej porušenia alebo z nekalosúťažného jednania \---} \href{https://eur-lex.europa.eu/legal-content/EN/TXT/?uri=CELEX%3A62014CJ0030}{Ryanair Ltd v PR Aviation BV.}

