\documentclass{article}
\usepackage[utf8]{inputenc}
\usepackage[czech]{babel}

\title{SPR - otázky 8 a 9}
\author{Vojtěch Schiller}
\date{January 2023}

\begin{document}

\maketitle

\section{Co je podstatou softwarových veřejných licencí a jak tyto fungují po právní stránce? Kdy zvolíte jakou veřejnou softwarovou licenci a proč?}

Veřejná licence je specifickým způsobem sjednaná licenční smlouva. SW licencovaný pod veřejnou licencí je vetšinou poskytován bez úplaty, tímto způsobem se lze zbavit odpovědnosti za chyby v programu, které nezpůsobují právní vady. Obsahuje podmínku uvedení autora.

Podstatou veřejné licence je zveřejnění díla s licenčními podmínkami, odkazem na ně. Kde
nabyvatel licence \textbf{není v přímém kontaktu} s poskytovatelem. a využívá se hlavně v situaci
kdy licenci chceme směřovat na \textbf{neurčitý počet osob}.

\uv{\textit{Veřejné licence jsou veřejné návrhy
k uzavření licenčních smluv, jejichž obsah je standardizován a vymezen odkazem na veřejně
známé a dostupné licenční podmínky a určen neurčitému počtu osob}}.

Nejčastěji se veřejných licencí využívá ve FOSS (Free Open Source Software). Typy licencí mohou být \textbf{silně copyleftové},
\textbf{slabě copyleftové} a \textbf{necopyleftové}.

\textbf{Silně copyleftové} nesou omezené při zpracovaní a šíření SW. Požadují, aby původní,
nebo nový, program, který obsahuje původní, byl šířen pod původními licenčními podmínkami a současně garantují tvůrci přístup ke zdrojovému kódu. Zástupci jsou GNU GPL v2 a v3 a EUPL.

\textbf{Slabě copyleftové} vyžadují šíření odvozených programů pod stejnými licenčními podmínkami a zpřístupnění jejich zdrojových kódů. Umožňují vytváření programů, které jsou
propojené a šířené společně s původním programem aniž by měnily či používaly jeho
zdrojový kód a tyto programy šířit pod libovolnou licencí. Nejčastěji to jsou standardní
knihovny. 

Nemusí se vydat zdrojové kódy vlastního kódu ale pouze musí uvést a zpřístupnit původní část programu pod původní licencí. Zástupci MPL (Mozila Public License) v1.1 a LGPL (Lesser General Public License) v2.1.

Na rozdíl od GPL není MPL \uv{virová}: program pod MPL lze kombinovat s nesvobodným softwarem, pouze převzatá část musí nadále splňovat podmínky MPL. (Při začlenění cizí tvorby uveřejněné pod GPL je nutno uvolnit pod GPL celý program.)

\textbf{Necopyleftové/Permisivní licence} neobsahují žádnou nebo velmi omezenou copyleftovou doložku.
Ukládají pouze minimální omezení k dalšímu šíření. Proto lze použít i při vývoji SW s neveřejným zdrojovým kódem aniž by bylo porušeno původních podmínek. Zástupci Apache 2.0, BSD a MIT.

Licence lze měnit směrem od nejslabší po nejsilnější ale ne naopak. Další často používanou
licencí je Creative Commons -- umožňuje přidat požadavky, jak zpracovávat (nepoužívat komerčně, nezpracovávat a uvést původ).

\section{Definujte správce osobních údajů a popište jeho základní povinnosti dle GDPR. Jaký rozdíl mezi správcem a zpracovatelem osobních údajů?}
Správce po většinou chce sbírat osobní údaje a sbírá je za předem definovaným účelem.
Lze mít více správců na jedny data aka každý odpovídá sám za sebe.

Zpracovatel je osoba/firma, která je najata správcem osobních údajů. Zpracovatel nemusí vždy existovat nebo jich může být více. Například je možné, že se osobní údaje
nezpracovávají, nebo si je správce zpracovává sám. Pokud zpracovatel začne rozhodovat
o účelu dat sám, stává se správcem. Mezi těmito entitami musí při zpracovaní být vždy
sepsána písemná smlouva.

Příklad: Máme firmu A, která prodává zboží a sbírá osobní údaje jako např. datum narození za účelem zkoumání věkového průměru jejích zákazníků. Tím
se firma a se stává správcem osobních údajů. Firma A zadá zpracování (úpravu, třídění, uspořádání, ...) firmě B nebo samostatné osobě, aby jí data zpracovala a sepíší spolu smlouvu. Firma B se tím pádem stává zpracovatelem těchto osobních údajů.

Accountability (Performativní pravidlo):
\uv{Nastav své zpracování tak, aby odpovídalo potřebám konkrétní situace, abys zpracovával osobní údaje korektně a férově.}

Povinný subjekt za něco někomu aktivně odpovídá
\begin{itemize}
    \item DPA (Data Protection Authority)
    \item Národní úřad pro ochranu osobních údajů
    \item Evropský sbor ochrany osobních údajů
\end{itemize}
Správce
\begin{itemize}
    \item = fyzická (FO) nebo právnická osoba (PO) která sama nebo společně s jinými určuje účely a prostředky zpracování osobních údajů
    \item nese \textbf{odpovědnost} (kterou má ze zákona povinnou u soudu \textbf{doložit} -- nutnost vést si dokumentaci)
    \item správce může pověřit zpracovatele
    \item Důležitý je účel zpracování
    \begin{itemize}
        \item nutno údaje hned smazat po splnění účelu
        \item pro jiné účely se údaje nesmí využít
        \item každé zpracování je nutné odůvodnit
        \item pod jeden pr. titul nelze zahrnout víc účelů (tabulka účelů a k nim odpovídajícím pr. titulů)
    \end{itemize}
\end{itemize}
Povinnosti správce:
\begin{itemize}
    \item Odpovědnost správce (doložitelná u soudu)
    \item Odpovědnost za zabezpečení zpracování
    \begin{itemize}
        \item Zavedení vhodných technických a organizačních opatření takových, aby byl schopen doložit, že je zpracování prováděno v souladu s nařízením
        \item Jedná se o přístup založený na riziku (Risk Based Approach) -- každý správce čelí jiné míře rizika a dle toho je nutné údaje zabezpečit
    \end{itemize}
    \item Záměrná a standardní ochrana osobních údajů
    \item Vedení záznamů o činnostech zpracování
    \item Spolupráce s dozorovým úřadem
    \item Nutnost hlásit případy porušení zabezpečení osobních údajů jak úřadu tak postiženým osobám
    \item Jmenování pověřence pro ochranu osobních údajů
    \item Posouzení vlivu na ochranu osobních údajů
    \begin{itemize}
        \item Pokud se předpokládá vysoké riziko pro práva a svobody osob
    \end{itemize}
\end{itemize}

Zpracovatel:
\begin{itemize}
    \item = FO nebo PO, která zpracovává osobní údaje pro správce
    \begin{itemize}
        \item Zpracování = Jakákoliv operace nebo soubor operací s osobními údaji (shromáždění, zaznamenání, uspořádání, strukturování, uložení, přizpůsobení nebo pozměnění, vyhledání, výmaz, seřazení, \dots)
        \item Zákonnost zpracování – musím mít nějaký (jeden a více) \textbf{zákonný} (právní) \textbf{titul}, který mi umožní zpracování
        \begin{enumerate}
            \item \textbf{Souhlas} se zpracováním
            \item Zpracování nezbytné pro \textbf{splnění právní povinnosti} správce
            \item Zpracování nezbytné pro \textbf{splnění smlouvy} (Např. adresa atd. při nákupu na internetu)
            \item \textbf{Ochrana životně důležitých zájmů} subjektu údajů (souhlas bez zbytečného odkladu) (např. podání krevní transfuze za účelem záchrany života)
            \item Nezbytnost pro \textbf{ochranu práv a oprávněných zájmů} správce či třetí osoby (rozhodování testem proporcionality)
            \item Zpracování je nezbytné pro \textbf{splnění úkolu} prováděného ve veřejném zájmu nebo při výkonu veřejné moci, kterým je pověřen správce
        \end{enumerate}
    \end{itemize}
    \item Jmenuje ho správce
    \item Může být víc správců na jednoho pověřence
    \item Většinou nezávislá osoba vůči správci nebo jeho organizaci
    \item Zpracovává jménem správce
    \item musí mít smlouvu a oprávnění v rozsahu pověření
\end{itemize}

\end{document}
